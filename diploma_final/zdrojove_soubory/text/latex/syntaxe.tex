%% History:
% Pavel Tvrdik (26.12.2004)
%  + initial version for PhD Report
%
% Daniel Sykora (27.01.2005)
%
% Michal Valenta (3.12.2008)
% rada zmen ve~formatovani (diky M. Duškovi, J. Holubovi a~J. Žďárkovi)
% sjednoceni zdrojoveho kodu pro~anglickou, ceskou, bakalarskou a~diplomovou praci

% One-page layout: (proof-)reading on displa
%%%% \documentclass[11pt,oneside,a4paper]{book}
% Two-page layout: final printing
%\documentclass[11pt,twoside,a4paper]{book}   
\documentclass{article}
\clubpenalty 10000
\widowpenalty 10000
\usepackage{graphicx}



%=-=-=-=-=-=-=-=-=-=-=-=--=%
% The user of this template may find useful to have an alternative to these 
% officially suggested packages:
\usepackage[czech, english]{babel}
%\usepackage[T1]{fontenc} % pouzije EC fonty 
% pripadne pisete-li cesky, pak lze zkusit take:
\usepackage[T1]{fontenc}
\usepackage{lmodern}
%\usepackage[cp1250]{inputenc}
\usepackage[utf8x]{inputenc}

\usepackage{listings}
\lstset{
  literate={ě}{{\v{e}}}1
           {š}{{\v{s}}}1
           {č}{{\v{c}}}1
           {ř}{{\v{r}}}1
           {ž}{{\v{z}}}1
           {ý}{{\'y}}1
           {á}{{\'a}}1
           {í}{{\'i}}1
           {é}{{\'e}}1
           {ó}{{\'o}}1
}


% Pozicovani
\usepackage{float}
\restylefloat{table}

%=-=-=-=-=-=-=-=-=-=-=-=--=%
% In case of problems with PDF fonts, one may try to uncomment this line:
%\usepackage{lmodern}
%=-=-=-=-=-=-=-=-=-=-=-=--=%
%=-=-=-=-=-=-=-=-=-=-=-=--=%
% Depending on your particular TeX distribution and version of conversion tools 
% (dvips/dvipdf/ps2pdf), some (advanced | desperate) users may prefer to use 
% different settings.
% Please uncomment the following style and use your CSLaTeX (cslatex/pdfcslatex) 
% to process your work. Note however, this file is in UTF-8 and a~conversion to 
% your native encoding may be required. Some settings below depend on babel 
% macros and should also be modified. See \selectlanguage \iflanguage.
%\usepackage{czech}  %%%%%\usepackage[T1]{czech} %%%%[IL2] [T1] [OT1]
%=-=-=-=-=-=-=-=-=-=-=-=--=%

%%%%%%%%%%%%%%%%%%%%%%%%%%%%%%%%%%%%%%%
% Styles required in your work follow %
%%%%%%%%%%%%%%%%%%%%%%%%%%%%%%%%%%%%%%%
\usepackage{graphicx}
%\usepackage{indentfirst} %1. odstavec jako v~cestine.

\usepackage{k336_thesis_macros} % specialni makra pro~formatovani DP a~BP



% Extension posted by Petr Dlouhy in order for better sources reference (\cite{} command) especially in Czech.
% April 2010
% See comment over \thebibliography command for details.

\usepackage[square, numbers]{natbib}             % sazba pouzite literatury
%\DeclareUrlCommandDostupné z \url{\def\UrlLeft{<}\def\UrlRight{>}\urlstyle{tt}}  %rm/sf/tt
%\renewcommand{\emph}[1]{\textsl{#1}}    % melo by byt kurziva nebo sklonene,
%\let\oldUrl\url
%\renewcommand\url[1]{<\texttt{\oldUrl{#1}}>}


\usepackage{fancyhdr}

\begin{document}
\selectlanguage{czech} 
\title{Použití CQML knihovny}
\date{}
\maketitle


Uživatel musí definovat uživatelské rozhraní v jazyce \textit{CQML} a uložit jej do souboru. Následně musí daný soubor předat překladači, který z něj vygeneruje zdrojové kódy pro~použití v uživatelské aplikaci, která pro obsluhu \textit{GUI} používá runtime knihovnu pro \textit{CQML}.
\section{Použití překladače}
\subsection{Spuštění překladu}
Překladač v příkazové řádce přijímá jeden parametr, kterým je jméno souboru, v němž je definována vrchní komponenta \textit{GUI} hierarchie. Ukázka kódu \ref{lst:help1} ilustruje spuštění překladače pro překlad souboru \uv{src.cqml} prostřednictvím příkazové řádky. V~případě, že jsou v rámci daného souboru využívány komponenty z jiných souborů, tyto soubory se načtou automaticky.
\begin{lstlisting}[frame=single,caption=Spuštění překladače pomocí příkazové řádky.,label=lst:help1]
parser.exe "src.cqml"
\end{lstlisting}
\subsection{Výstupní soubory}
Výstupem překladče je několik souborů ($X$ je pořadové číslo zpracovaného souboru):
\begin{description}
\item[parser\_output.cpp] - zdrojový soubor s vygenerovanými funkcemi pro inicializaci a obsluhu rozhraní.
\item[parser\_output.h] - hlavičkový soubor, v němž jsou deklarovány hlavičky funkcí pro inicializaci a obsluhu rozhraní.
\item[outputX.cpp] - zdrojový soubor komponenty.
\item[outputX.h] - hlavičkový soubor komponenty.
\end{description}
\section[Použití knihovny]{Použití knihovny v uživatelské aplikaci}
\subsection{Kompilace uživatelské aplikace}
Pro překlad uživatelské aplikace používající zdrojové kódy vygenerované překladačem je nutné použít dynamickou knihovnu, která byla zkompilována do souborů \uv{CQML\_DLL.dll} a \uv{CQML\_DLL.dll}. Do~projektu pro uživatelskou aplikaci je potřeba zahrnout soubory vygenerované překladačem a také hlavičkové soubory pro použití knihovny ze složky \uv{cqml\_include}.
\subsection{Inicializace rozhraní}
Před spuštěním hlavní programové smyčky, která komunikuje s \textit{GUI}, musí být provedena inicializace. Nejprve program musí zavolat funkci \textit{\_CQML\_Init}. Následně je nutné přiřadit instance rozhraní pro vykreslování a správu externích zdrojů pomocí funkcí \textit{SetDrawIFace} (viz. sekce \ref{SEC:drawI}) a \textit{SetResourceManager}(viz. sekce \ref{SEC:resI}). Na závěr je potřeba inicializaci ukončit zavoláním funkce \textit{\_CQML\_Start}, která spustí logiku pro obsluhu \textit{GUI}. Pro volání všech funkcí pro obsluhu \textit{GUI} musí být do zdrojového souboru zahrnut soubor \uv{qml\_includes.h} ze složky \uv{cqml\_include} a také soubor \uv{parser\_output.h} vygenerovaný překladačem.
\subsection{Zpracování vstupu}
Vstup se zpracovává prostřednictvím událostí, které musí být ve specifickém formátu předány knihovně prostřednictvím funkce \textit{PushEvent}. Tato funkce přijímá jeden parametr typu \textit{CQMLEvent}, ve kterém jsou uložena specifika dané vstupní události. Struktura datového typu \textit{CQMLEvent} je uvedena v kódu \ref{lst:help2}.
\begin{lstlisting}[frame=single,caption=Struktura datového typu \textit{CQMLEvent}.,label=lst:help2]
int EventType; //hodnota CQML_KEY_EVENT nebo CQML_MOUSE_EVENT
// následující hodnoty platí jen pro CQML_KEY_EVENT
	int keyEvent.action;	// KEY_RELEASED nebo KEY_PRESSED
	int keyEvent.key;	// kód stisknuté klávesy
// následující hodnoty platí jen pro CQML_MOUSE_EVENT
	int mouseEvent.action; // MOUSE_WHEEL_SCROLLED, 
		// MOUSE_MOVEMENT, MOUSE_BUTTON_RELEASED 
		// nebo MOUSE_BUTTON_PRESSED
	int mouseEvent.button; // kód stisknutého tlačítka
	int mouseEvent.x;	// souřadnice x ukazatele myši
	int mouseEvent.y;	// souřadnice y ukazatele myši
	int mouseEvent.relativeX; // relativní pohyb v ose x
	int mouseEvent.relativeY; // relativní pohyb v ose y
\end{lstlisting}
Kód \ref{lst:help3} ilustruje předání události stisknutí klávesy knihovně.
\begin{lstlisting}[float,frame=single,caption=Předání vstupu z klávesnice knihovně.,label=lst:help3]
CQMLEvent e;
e.EventType =  CQML_KEY_EVENT;
e.keyEvent.action = KEY_PRESSED;
e.keyEvent.key = c; // c je hodnota označující klávesu
PushEvent(e);
\end{lstlisting}

\subsection{\label{SEC:resI}Rozhraní pro zpracování externích zdrojů}
Pomocí funkce \textit{SetResourceManager} se nastaví rozhraní pro správu externích zdrojů, to je jí předáno prostřednictvím ukazatele na instanci daného rozhraní. Rozhraní musí dědit z~abstraktní třídy \textit{ResourceManagerIFace} a implementovat metody ukázané v kódu \ref{lst:help5}.
\begin{lstlisting}[frame=single,caption=Metody rozhraní pro správu externích zdrojů.,label=lst:help5]
void* LoadFont(const char * fontStr, int fontSize);
// načte data písma a vrátí na ně ukazatel
// fontStr je název písma
// fontSize je velikost písma

ImageData LoadImage(const char *  path);
// načte data obrázku a vrátí je ve formě struktury popsané níže
// path je cesta k souboru s obrázkem

struct ImageData
{
	void * img; // ukazatel na načtená data
	int w; // šířka načteného obrázku
	int h; // výška načteného obrázku
};
\end{lstlisting}

\subsection{\label{SEC:drawI}Vykreslovací rozhraní}
Pomocí funkce \textit{SetDrawIFace} se nastaví rozhraní pro vykreslování, to je jí předáno prostřednictvím ukazatele na instanci daného rozhraní. Rozhraní musí dědit z abstraktní třídy \textit{DrawIFace} a implementovat metody ukázané v kódu \ref{lst:help4}.
\begin{lstlisting}[float,frame=single,caption=Metody vykreslovacího rozhraní.,label=lst:help4]
void DrawFilledRectangle(int x,int y, int w, int h,
	float r, float g, float b);
// vykreslí jednobarevný obdélník
// x, y jsou souřadnice obdélníku
// w, h jsou rozměry obdélníku
// r, g, b určují barevné složky barvy výplně obdélníku

void DrawFilledBorderedRectangle(int x, int y, int w, int h, 
	float r, float g, float b, float border, 
	float br, float bg, float bb);
// vykreslí obdélník s okrajem
// x, y jsou souřadnice obdélníku
// w, h jsou rozměry obdélníku
// r, g, b určují barevné složky barvy výplně obdélníku
// border určuje šířku okraje
// br, bg, bb určují barevné složky barvy okraje obdélníku

void DrawText(int x, int y, int w, int h, 
	const char* text, void* font, float r, float g, float b);
// vykreslí text
// x, y jsou souřadnice textu
// text je textový řetězec, který se má vykreslit
// font je ukazatel na data písma
// r, g, b určují barevné složky barvy písma

void DrawImage(int x, int y, int w, int h, void* image);
// vykreslí obrázek
// x, y jsou souřadnice obrázku
// w, h jsou rozměry, do kterých se obrázek roztáhne
// image je ukazatel na data obrázku

void DrawImageSegment(int x, int y, int w, int h,
	void* image, int iX, int iY, int iW, int iH);
// vykreslí určitý výřez obrázku
// x, y jsou souřadnice obrázku
// w, h jsou rozměry, do kterých se obrázek roztáhne
// image je ukazatel na data obrázku
// iX, iY jsou souřadnice výřezu ve zdrojovém obrázku
// iW, iH jsou rozměry výřezu ve zdrojovém obrázku
\end{lstlisting}

\subsection{\label{SEC:updateDraw}Funkce volané z hlavní programové smyčky}
Pro vykreslování \textit{GUI} uživatelská aplikace musí zavolat funkci \textit{\_CQML\_Draw}. Pro přepočítání hodnot jednotlivých elementů v hierarchii uživatelská aplikace musí zavolat funkci \textit{\_CQML\_Update}.


\section[Rozšiřování \textit{CQML}]{\label{CH:APD}Rošiřování jazyka \textit{CQML} o nové vestavěné typy a elementy.}
Jazyk \textit{CQML} lze rozšířit o další vestavěné typy a to pomocí speciálních maker v souboru \uv{struct\_definition\_macros.h}, toto však vyžaduje zásah do zdrojový kódů knihovny. Nejprve programátor vytvoří makro pro definici struktury daného typu a následně dané makro zaregistruje. Toto je ilustrováno v kódu \ref{lst:help5}, kde je přidán nový element jménem \textit{EnhancedElement}, který je odvozen z typu \textit{Element}. Veškerou funkčnost, která není poskytována již u ostatních elementů, musí však programátor implementovat v rámci zdrojového kódu knihovny.\\
\begin{lstlisting}[float, frame=single,caption=Registrace nového vestavěného datového typu.,label=lst:help5]
// makro STRUCT_ENHANCED vytvoří strukturu s následujicími členy:
// 1. atribut typu int jménem Attribute s výchozí hodnotou 0
// 2. atribut typu float jménem HiddenAttribute, který není 
	přístupný z CQML, s výchozí hodnotou 0
// 3. virtuální metoda jménem VirtualMethod s návratovým 
//	typem void a žádnými a parametry
// 4. metoda jménem Method s návratovým typem int 
//	a dvěma parametry typu int
// 5. ukazatel na funkci pro obsluhu nějaké události, který je
//	z CQML přístupný pod jménem EventHandler
//	a je inicializován na nulovou hodnotu. 
//	v tomto případě přijímá parametr typu QMLMouseEvent, 
//	který je v CQML dostupný pod názvem EVENT
#define STRUCT_ENHANCED(MF, F, M, ME, MEV) \
	F(int, Attribute, 0) \
	MF(float, HiddenAttribute, 0) \
	MEV(void, VirtualMethod) \
	ME(int, Method, int, int) \
	M(void, CustomEventHandler, 0, QMLMouseEvent EVENT) \

// následně se dané makro zaregistruje přidáním nového řádku 
// 	do makra REGISTRATION (viz. jeho poslední řádek).
// první argument na posledním řádku určuje název registrovaného
//	makra
// druhý argument na posledním řádku určuje jméno elementu
// poslední argument na posledním řádku určuje, z jaké struktury
//	nový prvek dědí - pro případ, že chceme vytvořit
//	hodnotový datový typ, který z ničeho nedědí, odstraní se 
//	poslední argument a makro MACRO3 se změní na MACRO2
#define REGISTRATION(MACRO2, MACRO2REF, MACRO3) \
	REGPRIMITIVE(int)\
	MACRO2(STRUCT_COLOR, Color) \
	MACRO2REF(STRUCT_ELEMENT, Element) \
	MACRO3(STRUCT_RECTANGLE, Rectangle, Element) \
	MACRO3(STRUCT_ENHANCED, EnhancedElement, Element) \ 
\end{lstlisting}
Upravený soubor \uv{struct\_definition\_macros.h} musí programátor vložit do projektů všech částí knihovny \textit{CQML}, tj. překladače, preprocesoru datových typů i runtime knihovny a také do projektu uživatelské aplikace. Následně musí zkompilovat a spustit preprocesor datových typů, který vygeneruje zdrojový soubor \uv{preparser\_output.cpp}, který musí být zahrnut do~zdrojových kódů knihovny, resp. je jím nutno přepsat ten již existující. Po té je nutné znovu zkompilovat překladač a runtime knihovnu. Poté už je možné nový datový typ použít.

\end{document}
